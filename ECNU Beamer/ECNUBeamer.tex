\documentclass[10pt,aspectratio=169,mathserif]{beamer}		
%字体为 10pt,长宽比为16:9,数学字体为 serif 风格

\usepackage{ecnu}
\usepackage{ctex}
\usepackage{amsmath,amsfonts,amssymb,bm}
\usepackage{color}
\usepackage{graphicx,hyperref,url}	

\beamertemplateballitem		%设置 Beamer 主题

\catcode`\。=\active
\newcommand{。}{.}   % 这两行可以重定义所有中文字符

%%%%----首页信息设置----%%%%
\title{基于空间感知的蓝牙飞控实现}
\subtitle{——单片机与距离检测模块的集成}			

% ———————— 作者栏 ———————— %

\author[Ziwei Zhang \\ Zhuoqian Guan]{ % 下角标读取的内容
  Ziwei Zhang \\ Zhuoqian Guan \\
  \medskip{
    \small{10235101526@stu.ecnu.edu.cn \\ 10235101529@stu.ecnu.edu.cn} % \maketitle 时具体的内容
  }
}

\institute[ECNU]{East China Normal University \\ Department Of Software Engineering }

\date{2024.06.28}

\begin{document}

\begin{frame}
\titlepage % 生成标题页
\end{frame}

\section{Outline}
\begin{frame}
\frametitle{Outline}
\tableofcontents
\end{frame}

\section{Introduction}
\begin{frame}
  \frametitle{Backgrounds}

\end{frame}

\section{Progress}
\begin{frame}
  \frametitle{Main Thought}

\end{frame}

\section{Conclusion}
\begin{frame}
  \frametitle{Thanks}

  \noindent\textbf{Sequence Tagging Loss\\}
\end{frame}

\section{References}
\begin{frame}{References}
\begin{thebibliography}{99}
\bibitem{zhao1} Xuezhe Ma and Eduard Hovy. (2016).\\
{\bf End-to-end Sequence Labeling via Bi-directional LSTM-CNNs-CRF.\\}
In Proceedings of the 54th Annual Meeting of the Association for Computational Linguistics, pages 1064–1074, Berlin, Germany, August 7-12, 2016.
\end{thebibliography}
\end{frame}

\end{document}